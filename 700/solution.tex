\documentclass{article}
\usepackage{CJKutf8}
\usepackage{listings}
\usepackage{amsmath}
\usepackage{ulem}
\usepackage{algorithm}
\usepackage{algorithmic}
\begin{document}
\begin{CJK}{UTF8}{gbsn}
\setlength{\parindent}{0pt}
\section{问题}

莱昂哈德·欧拉出生于$1707$年四月$15$日。\\

考虑序列 $1504170715041707n \bmod 4503599627370517$。\\

该序列中的某个元素被定义为欧拉币,当且仅当它严格小于之前已经定义的所有欧拉币。\\

例如,序列的第一项是$1504170715041707$,这是第一枚欧拉币。序列的第二项是$3008341430083414$,因为它大于$1504170715041707$,因此它不是欧拉币。然后,序列的第三项是$8912517754604$,比第一项要小,因此是一枚新的欧拉币。\\

因此,前$2$枚欧拉币之和为$1513083232796311$。\\

求所有欧拉币之和。

\section{解法}

\subsection{暴力}

一个最直观的想法是,我们可以暴力枚举每个 $n$ 来判断是否比之前的欧拉币要小,以此来判断当前是否是一个欧拉币。代码大概如下:

\begin{algorithm}
\caption{Calculate all euler coin}
\begin{algorithmic}
\STATE $n \gets 1$
\STATE $step \gets 1504170715041707$
\STATE $mod \gets 4503599627370517$
\STATE $last\_euler\_coin \gets mod$
\STATE $answer \gets 0$
\WHILE {$n \neq mod$}
	\IF {$n*step\%mod < last\_euler\_coin$}
		\STATE $last\_euler\_coin \gets n*step\%mod$
		\STATE $answer += last\_euler\_coin$
	\ENDIF
	\STATE $n \gets n+1$
\ENDWHILE
\end{algorithmic}
\end{algorithm}

但是很遗憾,这个过程的复杂度是 $O(mod)$ 的,也即:即使 $1$s 能运算 $1e8$ 次,也需要 $521$ 天。这远远超过了欧拉计划的 $1$ 分钟原则。下面是我自己的暴力代码的速度:\\
\thispagestyle{empty}
\begin{tabular}{|c|c|c|}
\hline  
$n$&euler\_coin&时间花费\\
\hline 
1&1504170715041707&1.3e-05\\
\hline
3&8912517754604&3.6e-05\\
\hline
506&2044785486369&4.5e-05\\
\hline
$\dots$&$\dots$&$\dots$\\
\hline
3732049906&10487287&2.08483\\
\hline
4015876927&10078122&2.23973\\
\hline
$\dots$&$\dots$&$\dots$\\
\hline
10260071389&1076492&5.668\\
\hline
10543898410&667327&5.82323\\
\hline
10827725431&258162&5.9797\\
\hline
21939277883&107159&12.0777\\
\hline
54990108218&63315&30.237\\
\hline
\end{tabular}\\\\

再之后的一个 euler-coin 的计算已经超过 $60$ s了。

\subsection{推导}

通过简单计算,我们能够得到 $gcd(1504170715041707, 4503599627370517) = 1$。所以一定有 $1$ 是 euler-coin (因为 $1504170715041707n \equiv 1 (\bmod 4503599627370517)$ 是有解的)。 \\

那么一个有趣且自然的想法是,我们可以枚举每个值,来判断这个值是否是euler-coin,怎么判断是不是 euler-coin 呢?\\

首先,对于 euler-coin = $1$ 来说,我们需要找到对应的 $n_1$,不难发现 $n_1 = \frac{1}{1504170715041707} \bmod 4503599627370517 = 3451657199285664$\\

然后,对于剩下的 $x$ ,怎么判断 $x$ 是否是 euler-coin 呢?我们只需要计算 $n_x = \frac{x}{1504170715041707} \bmod 4503599627370517$,然后和上一个 euler-coin 对比下标,就能知道这个是不是 euler-coin 了。\\

然而,这个想法只能处理比较小的 $x$。\\

所以,我们将这个想法和暴力想法结合起来,对于比较大的 $x$ 和 较小的$n$,我们暴力从小到大枚举 $n$ 来找到 euler-coin,反之,我们从小到大枚举 $x$ 来判断每个 $x$ 是不是 euler-coin。

\end{CJK}
\end{document}
